\documentclass[11pt]{article}
\usepackage{geometry}
\usepackage{parskip}

\usepackage{float}
\usepackage{graphicx}

% bib
\usepackage[
style=ieee,
time=24h,
dateabbrev=false,
dateusetime=true,
style=numeric,
]{biblatex}
\addbibresource{invest2.bib}
\usepackage{hyperref}

% citation
% check mark scheme
% check structure
% proofread
% word count



\begin{document}	
\section*{Evaluation of an article}
% Framing question
% Are the latest generation of climate models useful for understnading future climate change.


% rubric
% comprehending scientific studies
% 	identify, explained, and discussed all the key concepts and arguments.
% 	aim, approach, and conclusions are identified and explained

% evaluating scientific methods and conclusions
% 	identify explain, and evaluated (pros and cons )scientific methods.
% 	analysis of the (correctness) of the conclusion drawn, and explained.

% ethical and culture considerations evaluation (of the resarch)
% 	demonstrate and outline ethical and culture concerns
% 	identify and explain the ethical issues in research and implications
% 	identity the culture context and assumptions the research is under, explain how it can influencen research method and findings



% evaluation of knowledge contribution (of research) to sustainability and climate change.
% identify and explain how the research contributes to understanding and action towards sus and climate change.
% pros and cons of this contribution identified and explained
% analysis on the social and environment impacts from the research, an implementation's feasibility is included.


% written expression, 
% citation, clarity


% article


% structure
% introduction
% 	summarize the contribution (of climate models) to future climate change, prediction, modelling.
%   consider the Aims, approach, and conclusions of the paper


% TODO: 923 WORDS

The article, ``Yes, a few climate models give unexpected predictions –-- but the technology remains a powerful tool'' is a scientific response to critics of climate models in the then-upcoming IPCC report\autocite{article}. Through its explanations of climate models, climate sensitivities, and IPCC projections, the authors formulate several evidence-based arguments in favor of future climate models and asserted the model's high accuracy and methods of error correction. However, the weak response to the criticisms of errors in new climate models, unbalanced considerations in its justifications, and an overly idealistic future outlook damage the text's rigor and conclusion.

% Typical format
% Key arguments, Methods & Results, Conclusions,
% Does argument hold? Are there gaps? Does the logic follow?
% Strongest and weakest argument points (in the article)

% Does the study have an overall aim/question
% Is it clear that the evidence relates to the aim?

% Has the conclusion been overstated based on evidence?
% Do the authors consider/identify errors and show caustion?

 
% Explain the response
% Discuss response
% Evaluate (Pros and Cons)

\begin{figure}[H]
	\centering
	\includegraphics[width=0.8\textwidth]{warming.png}
	\caption{Historical Climate Model Accuracy\autocite{hausfather_2017}}
	\label{fig:warming}
\end{figure}

% Response
The authors used an evidence-based approach in answering overestimation errors in their climate models. In response to ``So should we be using climate models?'', the key points made were that ``We know when the climate model fails'', with ``errors in some climate models ... means our understanding of the climate system has improved'', signaling that the errors of new climate models are not only detectable but also a sign of progress. This is supported by empirical evidence in the historical data on the reasonable accuracy of past climate models in predicting future warming (figure \ref{fig:warming})\autocite{hausfather_2017}. Moreover, the authors attributed their detection of misperforming climate models to the historical climate sensitivity data --- a model relating global warming with atmospheric CO2 levels --- where climate model outputs are first compared with climate sensitivity projections. The authors then identified incorrect model projections suggesting extremely low or extremely high-temperature changes, and applied the scientific method in coming up with a hypothesis --- in this case, the lack of clouds in the simulations\autocite{climate_2020} --- to explain and eventually correct the models. It supports the idea that climate model errors are not to be scared about, for their outputs are not bluntly presented without serious considerations that otherwise could misinform the vulnerable public. ``Scientists do not use climate models in isolation''.

% Climate sensitivity & projections
Yet, the authors' arguments contain inherent flaws and biases. While the historical data and the scientific process are from reputable sources, the authors have not directly addressed the concerns of the critics, that regarding the climate model's failures in overestimating global warming. The argument of having climate experts manually spot models ``running hot'' fails to dismiss the concerns facing undetected erroneous models and their potential influences on existing climate policies. Moreover, model accuracy from the past does not logically imply future accuracy in today's models, for it's a case of extrapolation. The models could be cherry-picked, or that simulation conditions at the current climate don't transfer to conditions in climates in the future --- two factors that were not considered in the article. Lastly, the use of climate sensitivities to detect anomalies contains a problematic feedback loop. If we view climate sensitivities to be a determinant of climate model correctness, we are simply optimizing our climate models to exactly the assumptions of our climate sensitivities, which at best contributes little extra information to our climate, at worse reinforces a potential error in the climate sensitivity data. This approach has a systematic risk of overstating or understating the climate risk we face, where the climate models may reinforce a bias in historic data that ultimately propagates an overly or under-aggressive signal guiding our climate policies.

% future outlooks
In its advocacy for the uses of climate models in the future, the authors explored the benefits of existing large-scale IPCC SSP and the potential fine-scale projections. It explains the SSPs to be a series of climate model projections associated with hypothetical socioeconomic models acting as initial conditions\autocite{RIAHI2017153}, of which it concludes ``considerable value in knowing both the future risks to avoid and what's possible under ambitious climate action''. The authors defended against the critique that the SSP high emission scenarios are too pessimistic, through a pragmatic view of the underdeveloped carbon capturing technology suggested overly optimistic low emission scenarios\autocite{extremes_2021}. At the fine scale, the authors urged the development of accurate climate models to assist with the fine-scale planning in forest plantings, flood defenses, and further environmental preserving decisions. In its conclusions, a sense of certainty and reliance towards climate models are seen in phrases like ``Climate models will continue to be an important tool ... to manage the unavoidable risk ahead'', and ``Climate models are already phenomenal tools at large scales''.

But its arguments are weak. Its response to the pessimistic high-emission model does not logically follow, for an optimistic low-emission scenario doesn't justify a pessimistic high-emission scenario if the aim is to model reality. Additionally, the suggested SSP fails to account for the unequal impacts of climate change regarding different regions and income classes. This has the potential ethical downside of misguiding readers into believing a better future in their circumstances, creating biases in climate policies that under-protects locations with high warming impacts.

% conclusion
In summary, while the points in the article are explained and well-supported, its rebuttals are logically weak with conclusions made not considering both sides of the argument. Its statement of climate model errors as learning opportunities follows the scientific method, but falls short as it fails to account for the risk in extrapolation; its perspective on the informational benefits of the SSP model has a logical basis but fails to consider the inequalities in the world. Climate models, ultimately, are a high-risk, high-reward asset that must be used with caution.

% main section
% 	strength and weaknesses of the article.
%   focus on the article's key arguments, methods and results, conclusions.

% conclusion/summary
% summarize my evaluation of the article, using the strength and weakeness identitified in the article.
\newpage
\printbibliography



\end{document}