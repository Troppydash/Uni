\documentclass[11pt]{article}

\usepackage{parskip}
\usepackage{geometry}



\linespread{1.213}

\begin{document}

% WORD COUNT = 867

\section*{Assignment 1}


% Client assessment
% Married, retirement planning advice
% aged 65, sole members of Self-managed super fund (in contrast to an industry fund)
% total of 700,000 accumulated in the fund, including equity and cash
% maintained approximate 50-50 split between the two (no idea why) for many years
% current asset holdings at the date of meeting is, full franking (no taxation)
% owns home (no mortgage)
% 130000 in lifestyle, non-investment assets
% two married adults and three grand-children, no liabilities

\subsection*{1}
% Clearly explain your obligations as a financial planner that would apply if the clients were to engage you to prepare a SOA.

% FROZEN
% 288
To prepare a Statement of Advice (SoA) is to provide personal advice, for it concerns the couple's personal circumstances --- who are retail clients without experienced investor status. My obligation would be to follow the statutory fiduciary law of financial planners, a legal duty introduced by the Future of Financial Advice Act. The summary is to act in the best interest of the client, where financial advice cannot benefit the planner without also benefiting the client.

To meet the fiduciary duty, I would need to fulfill the two acts of ``Know your client'' (KYC) and ``Know your product'' (KYP). KYC refers to understanding the client's circumstances, risks, and objectives, areas I'm obligated to perform a detailed analysis of and record afterward. KYP refers to my need to understand the fundamentals of financial products, with knowledge of their abilities and limitations in said instruments. KYC and KYP forms the suitability rule that acts as prerequisites for me to legally provide quality personal advice in a SoA, along with the requirement to act efficiently, honestly, and fairly in general.

Regarding the SoA, I must include sections required by the ASIC. Most important is the client's information containing their financial information: assets, existing and projected cash flows, risk profile, expectations and objectives; and personal information: name, address and details, preferences, and family circumstances. I'm required to obtain and sign this recorded information off and use it as a basis for any given financial advice. Moreover, I'm obligated to state any conflicts of interest in financial products and warn if my recommendations are based on incomplete data. I'm also required to list any fees and operate any implementation only after informing the client and getting their written consent to do so.


% Personal advice
% Fiducary law, client's best interest
% Suitability Rule: KYC, KYP
% Informed consent
% Items in the SOA
%   The advice
%   The basis of advice
%   Names and details of advisor (client?)
%   Conflict of interest
%   Warning of insufficient information
%   Alternatives
%   Client information
%   Fee information


% don't just copy and paste, and write in your own words
% no need to reference acts



\newpage
\subsection*{2}
% You have determined that the clients have not provided sufficient information to enable you to provide financial advice that meets regulatory requirements. Identify and explain what further information you would seek from them to enable you to prepare a SoA.

% While the couple detailed their asset holdings and light intents of retirement planning, more personal and financial information is required to satisfy my fiduciary duty --- particularly KYC --- to produce a SoA.

% FROZEN
% 296
To fulfill the regulatory requirements of the SoA, I additionally require a mixture of personal and financial information from the couple.

To understand the couple's personal circumstances and wishes, I'd seek a mix of basic information, retirement planning objectives, projected lifespan, and their investment and lifestyle preferences. First would be the couple's names, addresses, and contact details. Then would be their retirement objectives, used as a basis for my wealth investment judgments. This may include a simple RIS generation matching their current consumption, or optionally the inclusion of planned trips and major expenditures. I'd also query the couple's intention of their super-fund between a lump sum or a regular pension. In either case, I'd seek the couple's preferred investment horizon and estimated longevity, to best gauge the time aspect of their retirement expenditures. These questions regarding their investment and personal objectives lend me information on how to act in their best interest.

To understand the couple's financial circumstances to the level required by regulations, I'd seek a mix of their current liabilities and debt, additional personal investments, projected after-retirement work-related non-investment income, and risk profile. While I have a detailed description of their current assets, I require their current and expected debt undertakings to construct a fuller picture of their wealth situation for risk assessment and portfolio construction. The query on the existence and magnitude of personal investments achieves a similar goal. It's also for a better judgment of the available retirement consumption that prompts me to query their projected non-investment income from work or royalties. Lastly, I'd ask for all past equity investments and an explanation for their historical asset allocation strategy, for they form the basis of the client's risk profile required for the SoA.

% Information required
% Basic information, name, address, contact
% intent, objectives, preferences in investments
% projected lifespan, investment horizon
% current and projected expenditure

% family circumstances, family related cash flows
% existing liabilities, additional personal investments
% current income (if not retired yet), expected non-investment earnings
% risk profile from historical investments

\newpage
\subsection*{3}
%AA optimal, perceived risk aversion, more formal assessment.

% Give your opinion as to whether the current asset allocation is optimal for the couple (noting the possibility that the portfolio might be difficult to improve upon). Explain the reasons for your opinion. In your explanation, address how you perceive the clients' degree of risk aversion, and how you would seek to assess this more formally

% Your opinion on whether the current AA is optimal for the couple
% It is reasonable to say that the security portfolio is near perfection and to change
% Explain and support your opinion
% Touch on the perceived level of risk aversion, and how one can assess this more systematically

% diversification, industry concentration, asset class,

% main idea
I view the couple's current asset allocation to be not optimal under their retirement intent and risk aversion. Their current portfolio with its high proportion of equity may be dangerous for retirement due to short-term volatility. Moreover, their equity portfolio can be improved through diversification and index funds with minimal effort.

% risk analysis
My calculations show an approximate $46\%$ cash and $54\%$ equity allocation under market prices, with a modest expected return of $5\%$\footnote{See Appendix B}. Along with the relatively conservative targeted ratio of $50$-$50$ in their past asset allocations, I view the couple's risk aversion to be low. This is further supported by their industry diversified equities assets, and MP1 --- the portfolio's systematically riskiest security\footnote{See Appendix A, for its high beta of $1.03$} --- having the smallest weighting of $3.1\%$. The short-term cash deposit also hints towards risk aversion due to its low reinvestment and interest rate risk. For a more formal assessment of their risk profile, I'd conduct a risk assessment questionnaire and use their historical financial behavior by inquiring about the couple's past equity transactions and debt levels.


On the macro level, the couple's asset allocation can be improved through a more conservative allocation. Due to their low perceived risk tolerance and retirement age ranges, I propose a reduction in the equity allocation towards a more defensive level of 25\%. This reduces the impacts of market volatility on the couple's wealth, lowering the likelihood for an upcoming recession to wipe out their RIS generating assets. Similarly, I'd also retain their passive SAA strategy. For the stocks are fully franked, I see no tax trade-offs between stocks and cash investments. On the micro side, switching to an index ETF can optimize the diversification from a mere 5 stocks. This helps to both reducing unsystematic risk and improving cost efficiency by lowering brokerage fees in the future from asset class re-balances.

% details about asset allocation
% details about security selection


\newpage
\appendix
\section*{Appendix}
\section{Equity Portfolio}
Equity portfolio, beta is 5y statistic from Yahoo Finance

\begin{tabular}{|c|l|l|l|l|l|l|}
    \hline
    Ticker & Industry & Amount & Price & Stake & Weight & Beta \\
    \hline
    \hline
    ANZ & Banking & 3000 & \$24.9 & \$74700 & 0.196445829 & 0.83 \\
    \hline
    BHP & Materials & 3200 & \$43.15 & \$138080 & 0.363122358 & 0.83 \\
    \hline
    CSL & Healthcare & 400 & \$270.85 & \$108340 & 0.284912198 & 0.2 \\
    \hline
    WOW & Consumers Good & 1250 & \$37.75 & \$47187.5 & 0.124093542 & 0.23 \\
    \hline
    MP1 & Technology & 1000 & \$11.95 & \$11950 & 0.031426073 & 1.03 \\
    \hline
    Total &  &  & \$380257.5 &  & 0.58233 & \\
    \hline
\end{tabular}

Expected return on equity using CAPM, market expected return taken as 8\%, risk-free return taken from the couple's cash investment
\[
    E(r) = 0.0375 + 0.58233 (0.08 - 0.0375) = 0.062249
\]

\section{Total Portfolio}
Total portfolio

\begin{tabular}{|c|l|l|l|}
    \hline
    Types & Amount & Weights & Expected Returns \\
    \hline
    \hline
    Cash & \$325000 & 0.460824592 & 0.0375\\
    \hline
    Equity & \$380257.5 & 0.539175408 & 0.062249  \\
    \hline
    Total & \$705257.5 & & 0.05084 \\
    \hline
\end{tabular}


% calculations about the asset allocation ratios




\end{document}